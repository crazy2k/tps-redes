\newpage
\section{Discusión}
\subsection{Tiempo de transmisión}
\indent En el gráfico podemos observar como el tiempo de transmisión usando
ventanas de 1, 10 y 15, es muy similar. En cambio, cuando usamos una ventana de
20, es notable la diferencia en el tiempo que tarda en enviarse el
archivo. Esto se debe a la cantidad de timeouts que ocurren al utilizar este tamaño de ventana.\\

\subsection{Velocidad de Transferencia}
\indent Al igual que en el gráfico anterior, podemos observar la gran diferencia
que existe en la velocidad de transmisión cuando la ventana de emisión crece.\\
\indent La velocidad se mantiene estable con ventanas de 1,
10 y 15, pero cuando usamos una ventana de 20, la velocidad
tiende a caer mucho a medida de que aumenta el tamaño del archivo.\\
\indent Si comparamos este gráfico con el anterior, se ve claramente como al
ser menor el throughput, el tiempo de transmisión es mucho mayor.\\

\subsection{Cantidad de TimeOut}
\indent Al igual que en las 2 secciones anteriores, podemos observar la gran
diferencia que hay entre las ventanas, en este caso, con respecto a los
timeout.\\
\indent Vemos que con ventanas de 1, 10 y 15, no tuvimos timeout en ninguno de
los archivos que mandamos. En cambio, vemos que con la ventana en 20, los
archivos mas chicos tampoco tuvieron timeout, cuando los archivos mas grandes
empiezan a presentar cada vez mas timeouts. Los timeouts se dan ya que los paquetes encolados en el buffer del emisor se les acaba el ttl dado que los acks que se reciben del receptor no llegan lo suficientemente rápido para evitar que se acabe dicho tiempo para el paquete. Si seguimos aumentando el send window a tamaños mayores a 20 solo empeoraríamos la perfomance ya que al producirse un timeout se aplica una política de retransmisión de GoBackN lo que implica que una mayor cantidad de paquetes se deben retransmitir.