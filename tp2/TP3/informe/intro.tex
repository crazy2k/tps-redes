\newpage
\section{Introducción Teórica}
\subsection{PTC} 
\indent Extracto del enunciado del TP sobre que es el protocolo PTC.\\
\indent El protocolo que estudiaremos, PTC (Protocolo de Teoría de las
Comunicaciones), puede ubicarse dentro de la capa de transporte del modelo OSI
tradicional. Fue concebido como un protocolo de exclusivo uso didáctico que
permita lidiar en forma directa con algunas de las problemáticas usuales de la
capa de transporte: establecimiento y liberación de conexión, control de errores
y control de flujo.

\subsection{Capa de transporte}
\indent El nivel de transporte o capa de transporte es el cuarto nivel del
modelo OSI encargado de la transferencia libre de errores de los datos entre el
emisor y el receptor, aunque no estén directamente conectados, así como de
mantener el flujo de la red. Es la base de toda la jerarquía de protocolo. La
tarea de esta capa es proporcionar un transporte de datos confiable y económico
de la máquina de origen a la máquina destino, independientemente de la red de
redes física en uno. Sin la capa transporte, el concepto total de los protocolos
en capas tendría poco sentido.