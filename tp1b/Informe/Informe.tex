\documentclass[a4paper,spanish,12pt]{article}
\usepackage[spanish]{babel}
\usepackage[utf8]{inputenc}
\usepackage[pdftex]{graphicx}
\usepackage{vmargin}
%\usepackage{algorithmic}
%\usepackage{float}
\usepackage{lastpage}
\usepackage{caratula}
\usepackage{algorithm}
\usepackage{algorithmic}
\usepackage{url}


%%%%%%%%%%%%%%%%%%%%%%%%%%%%%%%%%%%%%%%%%
% ECO Para jugar con el footer. 		%
\usepackage{fancyhdr}					%
%%%%%%%%%%%%%%%%%%%%%%%%%%%%%%%%%%%%%%%%%


%%\usepackage{hyphenat}
%%\exhyphenpenalty=10000
%%\hyphenpenalty=10000
\setmarginsrb{10mm}% left margin
{15mm}% top margin
{15mm}% right margin
{10mm}% bottom margin
{0mm}{20mm}{0mm}{30mm}% we needed -- related to headers and footers

%%%%%%%%%%
\pagestyle{fancy}
\fancyhf{}

\fancyhead[RO, CE]{Teor\'{i}a de las Comunicaciones}
%\fancyhead[LO, CE]{Grupo X}
\fancyfoot[C]{ Página \thepage\ de \pageref{LastPage} }

\renewcommand{\headrulewidth}{0.6pt}
\renewcommand{\footrulewidth}{0.6pt}

\newcommand{\ite}[3]{\textbf{if} #1 \textbf{then} \\ \hspace*{7mm}\vbox{#2} \\ \textbf{else} \\ \hspace*{7mm}\vbox{#3} \\ \textbf{fi} }
\newcommand{\itf}[3]{\textbf{if} #1 \textbf{then} \\ \hspace*{7mm}\vbox{#2} \\ \textbf{fi} \\ }
\newcommand{\whi}[2]{\textbf{while (} #1 \textbf{)} \\ \hspace*{7mm}\vbox{\noindent #2} }
\newcommand{\tades}[2]{\noindent\textbf{TAD} #1 \textbf{ES} #2 \\}
\newcommand{\punt}[1]{($\ast$#1)}
\newcommand{\fle}{$\rightarrow$}


%%%%%%%%%%

\begin{document}


%*************************************************************%
%                                                             %
% CARATULA                                                    %
%                                                             %
%*************************************************************%
    \materia{Teor\'{i}a de las Comunicaciones}

    \titulo{Trabajo Práctico 1b}
  
    %\subtitulo{Tres problemas de programación}

    %\grupo{Grupo 255}

    \integrante{Antonio, Pablo}{290/08}{pabloa@gmail.com}
	\integrante{Ferrari, Gastón}{775/07}{gastonferrari5@hotmail.com}
    \maketitle
    
    \lhead{Trabajo Práctico 2}
    
%*************************************************************%
%                                                             %
% INFORME                                                     %
%                                                             %
%*************************************************************%

	\tableofcontents
	\newpage

\section{Introducción}
\indent En este trabajo practico nos dedicaremos a explorar el nivel de red. Para hacerlo, implementaremos la herramienta Traceroute, para luego usarla para explorar y conocer mas la red.\\
\indent También analizaremos lo compleja que es la red y las caracteristicas de la misma.

\subsection{Traceroute}
\indent Traceroute es una herramienta de diagnóstico que permite seguir la pista
de los paquetes que vienen desde un host (punto de red). Es decir, se envía un
paquete a una dirección IP y por cada router por el que pasa el paquete hasta
llegar al destino, se devuelve un paquete con la información de ese enlace. Asi,
es posible lograr saber por donde pasa un paquete hasta que llega a su destino
final. Se obtiene además una estadística del RTT o latencia de red de esos
paquetes, lo que viene a ser una estimación de la distancia a la que están los
extremos de la comunicación.

\subsection{ICMP}
\indent ICMP (Internet Control Message Protocol) es el sub protocolo de control
y notificación de errores del Protocolo de Internet (IP). Como tal, se usa para
enviar mensajes de error, indicando por ejemplo que un servicio determinado no
está disponible o que un router o host no puede ser localizado. Es importante
destacar que este protocolo no intecambia datos. Los paquetes ICMP viajan dentro
de los paquetes IP como si fueran de un protocolo de nivel superior. 



\newpage

\newpage

\section{Primera Parte: Estimación de RTT}
En esta primera parte del trabajo, el objetivo es realizar mediciones de
\emph{round-trip time} tanto teóricas como empíricas y razonar alrededor de
los resultados.

\subsection{Caracterización de los experimentos}

Elegimos tres universidades en distintas partes del mundo (Cambridge
University, Inglaterra; Stanford University, Estados Unidos; Moscow State
University, Rusia) y, tomando la dirección IP del dominio principal de cada
una, realizamos las siguientes mediciones:
\begin{enumerate}
    \item Cálculo del RTT teórico. Para esto, primero definimos la ubicación
        geográfica de la dirección haciendo uso de un servicio de
        geolocalización. Con ella, calculamos la distancia lineal desde ese
        punto hasta el que sería el origen de nuestras mediciones. Tomando un
        tiempo de propagación de la señal de $2 \times 10^5$ km/s, calculamos
        el tiempo que tardaría la ida de un paquete hasta el servidor destino
        y la llegada de una respuesta.
    \item Medición de RTT mediante una herramienta propia. Utilizando una
        herramienta de \emph{traceroute} desarrollada por nosotros usando la
        biblioteca Scapy, calculamos el RTT empírico a la dirección destino.
    \item Medición de RTT mediante una herramienta del sistema operativo.
        Idéntico al caso anterior pero esta vez usando el \emph{traceroute}
        disponible en una distribución de GNU/Linux.
    \item Medición de RTT teórico teniendo en cuenta el camino recorrido por
        los paquetes. Similar al cálculo teórico anterior, esta vez teniendo
        en cuenta las ubicaciones geográficas de cada uno de los routers
        intermedios.
\end{enumerate}

Previamente a las mediciones, se esperaba que el RTT teórico lineal fuera, en
general, muy menor a los obtenidos empíricamente, puesto que este, además de
dejar de lado retardos ocasionados por los routers en distintas horas del día,
asumía la existencia de un medio de comunicación (fibra óptica) cubriendo la
menor distancia entre el origen y el destino de la medición. Dado que el RTT
teórico del camino no asume esto último, se esperaba que este se ajustara más
a los valores empíricos. Por otro lado, se esperaba ver que los valores de RTT
empíricos arrojados por ambas herramientas de \emph{traceroute} fueran
cercanos, ya que las mediciones se realizarían en condiciones similares y los
métodos para realizar los cálculos serían similares.

Asimismo, como realizaríamos mediciones para distintos momentos del día, se
esperaba ver diferencias en los RTT según el horario, aunque no se suponía
ningún patrón en particular.

\subsection{Resultados de los experimentos}

Los gráficos \ref{fig:cambridge:count}, \ref{fig:stanford:count} y
\ref{fig:msu:count} muestran las mediciones para la Cambridge University
(Inglaterra), Stanford University (Estados Unidos) y la Moscow State
University (Rusia) respectivamente. En ellos pueden verse los valores para
distintos horarios del día.

\begin{figure}[h!]
    \centering
    \includegraphics[width=400pt]{cambridge.png}
    \caption{Cambridge University}
    \label{fig:cambridge:count}
\end{figure}

\begin{figure}[h!]
    \centering
    \includegraphics[width=400pt]{stanford.png}
    \caption{Stanford University}
    \label{fig:stanford:count}
\end{figure}

\begin{figure}[h!]
    \centering
    \includegraphics[width=400pt]{msu.png}
    \caption{Moscow State University}
    \label{fig:msu:count}
\end{figure}

\subsection{Análisis de los resultados}

\indent Nos pareció interesante saber cuantos km de mas tuvimos que recorrer dado que, por razones obvias, no tenemos un enlace punto a punto desde nuestras casas hacia el destino. Es por eso que están graficados ambos RTTs teóricos, creimos que tomando la distancia recorrida hasta llegar a destino iba a dar un RTT más ajustado a la realidad y así fue.\\

\indent Lo primero que se observa es como a la madrugada el RTT es menor en las dos rutas hacia Europa debido a que en ese horario el tráfico es menor tanto en el origen como en el destino, esto no sucede en la ruta hacia Estados Unidos ya que al ser en la costa oeste 4hs menos que en Buenos Aires a esa hora todavía se registra un tráfico mayor.\\

\indent En el primer gráfico llama la atención que el RTT teórico sea superior en algunos casos al RTT observado, creemos que esto se debe a un posible error en la herramienta de geolocalización ya que hay un desvío de unos 1000km en uno de los últimos hops. También puede estar sucediendo que ese hop, que tiene una IP de Escocia, en realidad sea un servidor que está en Inglaterra pero por alguna razón tiene una IP de ese país. Lo mismo sucede con uno de los primeros hops, que tiene una IP de Estados Unidos pero el RTT medido es de 28ms.\\

\indent Por último vimos que los resultados de las dos herramientas son bastante similares, probablemente el hecho de que la implementación de Traceroute del sistema operativo haga 3 intentos por hop sea la razón por la cual su curva sea más suave.\\



\section{Segunda Parte: Búsqueda de enlaces transatlánticos}
En esta segunda parte, el objetivo es encontrar los enlaces transatlánticos en
las rutas ya estudiadas en la primera parte.

\subsection{Caracterización de los experimentos}
Para esta segunda parte, modificamos nuestra herramienta de \emph{traceroute}
con el objetivo de que esta sea capaz de estimar cuáles de los enlaces pueden
ser transatlánticos. Para esto, se implementó la heurística sugerida que marca
un enlace como transatlántico cuando se cumple que:

$$r > R + m.d$$

siendo:
\begin{itemize}
    \item $r$ la diferencia entre los RTTs de ambas puntas del enlace,
    \item $R$ el promedio de las diferencias de los RTTs entre hops sucesivos
        durante la ejecución del \emph{traceroute},
    \item $d$ el desvío estándar de estas mediciones y
    \item $m$ fijado en 2.
\end{itemize}

\subsection{Resultados de los experimentos}
Lamentablemente en todas los experimentos que realizamos no pudimos identificar ningún enlace transatlántico real pero pudimos distinguir puntos en común entre los experimentos que nos dieron algunos indicios de por qué no los pudimos identificar.\\
Por ejemplo, en la ruta hacia Cambridge notamos un salto muy grande entre dos hops:\\

13 200.89.165.222  BUENOS AIRES, DISTRITO FEDERAL, ARGENTINA   Lat=-34.6132 Long=-58.3772   Time=46.5049743652 ms\\
14 208.178.244.125  BROOMFIELD, COLORADO, UNITED STATES   Lat=39.8828 Long=-105.106   Time=56.2880039215 ms\\
ENLACE TRANSATLANTICO\\
15 67.17.105.238  BROOMFIELD, COLORADO, UNITED STATES   Lat=39.8828 Long=-105.106   Time=161.536931992 ms\\

Claramente no puede haber tanta diferencia entre dos hops tan cerca uno de otro por lo que se nos ocurre que puede ser que haya un host en Argentina con una IP de Estados Unidos y que esté conectado con el siguiente hop mediante un túnel. Lo mismo vimos en las rutas hacia Stanford y Moscú.\\

Otra cosa que nos pareció importante destacar es que en las rutas hacia Cambridge y Moscú siempre apareció un hop que no respondía los mensajes de ping y éste se encontraba antes de un túnel que conectaba Estados Unidos con Europa, ésto es más que una suposición ya que en la ruta obtenida con el traceroute del sistema operativo se ven las etiquetas MPLS:\\
15 po4-40G.ar5.NYC1.gblx.net (67.17.105.238)  182 ms  170 ms po3-40G.ar5.NYC1.gblx.net (67.17.110.254)  206 ms\\
16  * * *\\
17  vlan80.csw3.NewYork1.Level3.net (4.69.155.190) [MPLS: Label 1948 Exp 0]  307 ms\\ vlan70.csw2.NewYork1.Level3.net (4.69.155.126) [MPLS: Label 1947 Exp 0]  270 ms vlan90.csw4.NewYork1.Level3.net (4.69.155.254) [MPLS: Label 1558 Exp 0]  293 ms\\
18  ae-61-61.ebr1.NewYork1.Level3.net (4.69.134.65) [MPLS: Label 1967 Exp 0]  286 ms ae-81-81.ebr1.NewYork1.Level3.net (4.69.134.73)  311 ms *\\
19  ae-41-41.ebr2.London1.Level3.net (4.69.137.65) [MPLS: Label 1638 Exp 0]  352 ms ae-43-43.ebr2.London1.Level3.net (4.69.137.73)  311 ms  558 ms\\
20  vlan101.ebr1.London1.Level3.net (4.69.143.85) [MPLS: Label 1496 Exp 0]  412 ms vlan102.ebr1.London1.Level3.net (4.69.143.89)  336 ms vlan103.ebr1.London1.Level3.net (4.69.143.93)  273 ms\\


\subsection{Conclusiones de los experimentos}
Dados los resultados podemos decir que la heurística propuesta tiene mucho sentido a nivel teórico pero en la práctica no es muy útil para descubrir enlaces transatlánticos.\\
Se nos ocurre que tal vez sería posible buscar los enlaces existentes y tratar de encontrar rutas que pasen por ellos, aún así sería dificil lograr localizar las IPs de los routers de ambas puntas de los enlaces ya que existen túneles que hacen casi imposible saber si se esta usando un enlace u otro.\\

 
\end{document}
