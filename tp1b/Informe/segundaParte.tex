\section{Segunda Parte: Búsqueda de enlaces transatlánticos}
En esta segunda parte, el objetivo es encontrar los enlaces transatlánticos en
las rutas ya estudiadas en la primera parte.

\subsection{Caracterización de los experimentos}
Para esta segunda parte, modificamos nuestra herramienta de \emph{traceroute}
con el objetivo de que esta sea capaz de estimar cuáles de los enlaces pueden
ser transatlánticos. Para esto, se implementó la heurística sugerida que marca
un enlace como transatlántico cuando se cumple que:

$$r > R + m.d$$

siendo:
\begin{itemize}
    \item $r$ la diferencia entre los RTTs de ambas puntas del enlace,
    \item $R$ el promedio de las diferencias de los RTTs entre hops sucesivos
        durante la ejecución del \emph{traceroute},
    \item $d$ el desvío estándar de estas mediciones y
    \item $m$ fijado en 2.
\end{itemize}

\subsection{Resultados de los experimentos}

\subsection{Análisis de los resultados}

