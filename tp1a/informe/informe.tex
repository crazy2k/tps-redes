\documentclass{article}
\usepackage[T1]{fontenc}
\usepackage[utf8]{inputenc}
\usepackage[spanish]{babel}
\usepackage{geometry}
\usepackage[conEntregas]{caratula}

\begin{document}

\newgeometry{left=2cm,right=2cm}

\titulo{Trabajo Práctico 1A: \emph{Wiretapping}}
%\subtitulo{Subtítulo del tp}

\fecha{\today}

\materia{Teoría de las Comunicaciones}
%\grupo{Grupo 42}

\integrante{Antonio, Pablo}{290/08}{pabloa@gmail.com}
\integrante{Apellido, Nombre2}{004/01}{email4@dominio.com}

\maketitle

\newgeometry{left=5cm,right=5cm}

\section{Introducción}
El presente informe corresponde al Trabajo Práctico 1A, titulado
"\emph{Wiretapping}", de la materia Teoría de las Comunicaciones. El objetivo
de este trabajo es desarrollar una herramienta sencilla de diagnóstico de red
y realizar un análisis a partir de la información que esta nos provee en
distintos segmentos de red.


\end{document}
