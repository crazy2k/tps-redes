\documentclass{article}
\usepackage[T1]{fontenc}
\usepackage[utf8]{inputenc}
\usepackage[spanish]{babel}
\usepackage[pdftex]{graphicx}
\usepackage{geometry}
\usepackage{caratula}

\begin{document}

\newgeometry{left=2cm,right=2cm}

\titulo{Trabajo Práctico 1A: \emph{Wiretapping}}
%\subtitulo{Subtítulo del tp}

%\fecha{\today}

\materia{Teoría de las Comunicaciones}
%\grupo{Grupo 42}

\integrante{Antonio, Pablo}{290/08}{pabloa@gmail.com}
\integrante{Ferrari, Gastón}{775/07}{gastonferrari5@hotmail.com}

\maketitle

\newgeometry{left=5cm,right=5cm}

\tableofcontents

\newpage

\section{Introducción}
El presente informe corresponde al Trabajo Práctico 1A, titulado
"\emph{Wiretapping}", de la materia Teoría de las Comunicaciones. El objetivo
de este trabajo es desarrollar una herramienta sencilla de diagnóstico de red
y realizar un análisis a partir de la información que esta nos provee en
distintos segmentos de red.

\section{Análisis de la entropía}
Para realizar este análisis, elegimos tres fuentes de información diferentes:

\begin{itemize}
    \item Fuente cuyos símbolos son IPs que realizan una consulta
        (\emph{request}) vía el protocolo ARP.
    \item Fuente cuyos símbolos son IPs que responden una consulta
        (\emph{reply}) vía el protocolo ARP.
    \item Fuente cuyos símbolos son IPs por las que se realiza una consulta
        (\emph{request}) vía el protocolo ARP.
    \item Fuente cuyos símbolos son IPs que a la que le responden una consulta
        (\emph{reply}) vía el protocolo ARP.
\end{itemize}

Los datos fueron tomados de redes hogareñas.

\subsection{Red 1}
\subsubsection{Símbolos: IPs que realizan una consulta ARP}
La siguiente tabla muestra, para cada símbolo, su cantidad de apariciones y
sus medidas de probabilidad e información asociadas.

\vskip10pt

\begin{tabular}{|l|l|l|l|l|}
  \hline
  Símbolo & Apariciones & Probabilidad & Información \\
  \hline
  192.168.0.105 & 99 & $0.0947368421053$ & $3.39993060689$ \\
  \hline
  192.168.0.104 & 28 & $0.0267942583732$ & $5.22193230491$ \\
  \hline
  192.168.0.1 & 415 & $0.397129186603$  & $1.33231970073$ \\
  \hline
  192.168.0.101 & 448 & $0.428708133971$ & $1.22193230491$ \\
  \hline
  192.168.0.100 & 1 & $0.000956937799043$ & $10.029287227$ \\
  \hline
  192.168.0.103 & 11 & $0.0105263157895$ & $6.56985560833$ \\
  \hline
  192.168.0.102 & 26 & $0.0248803827751$ & $5.32884750883$ \\
  \hline
  0.0.0.0 & 17 & $0.0162679425837$ & $5.94182438572$ \\
  \hline
\end{tabular}

\vskip10pt

La entropía de la fuente es:

$$H(S) = 1.82297065237$$

En la figura \ref{fig:red1requesters:infoentro}, puede observarse gráficamente
los valores de información de cada símbolo en relación con la entropía de la
red.

\begin{figure}[h!]
    \centering                                                       
    \includegraphics[width=300pt]{consultas1.png}
    \caption{Información de cada símbolo, en relación con la
        entropía de la red}
    \label{fig:red1requesters:infoentro}
\end{figure}

La figura \ref{fig:red1requesters:count} muestra la cantidad de apariciones de
cada símbolo. Podemos observar que las IPs 192.168.0.1 y 192.168.0.101 son las
que más pedidos realizan.

\begin{figure}[h!]
    \centering                                                       
    \includegraphics[width=300pt]{red1requesters.png}
    \caption{Apariciones de los símbolos}
    \label{fig:red1requesters:count}
\end{figure}

La figura \ref{fig:red1requesters:graph} permite observar qué IPs realizaron
consultas sobre qué otras IPs de la red. El tamaño de cada nodo se corresponde
con la cantidad de pedidos que realizó cada IP.

\begin{figure}[h!]
    \centering
    \includegraphics[width=350pt]{red1requestersgraph.png}
    \caption{Grafo en el que hay un eje desde una IP A a una IP B si la IP A
        realizó una consulta por la IP B}
    \label{fig:red1requesters:graph}
\end{figure}


Claramente se pueden distinguir dos nodos:
192.168.0.1 que es el router, \'esto es l\'ogico considerando que \'el es el encargado de distribuir todo el tr\'afico de la red a los nodos correspondientes por lo que su tabla
de direcciones macs tiene que estar correctamente actualizada el mayor tiempo
posible.\\
192.168.0.101 que es un host con un disco que comparte con el resto de la red.\\
\newpage

\subsubsection{Símbolos: IPs que responden una consulta ARP}
La siguiente tabla muestra, para cada símbolo, su cantidad de apariciones y
sus medidas de probabilidad e información asociadas.

\vskip10pt

\begin{tabular}{|l|l|l|l|l|}
  \hline
  Símbolo & Apariciones & Probabilidad & Información \\
  \hline
  192.168.0.105 & 506 & $0.85472972973$ & $0.226459790935$ \\
  \hline
  192.168.0.101 & 24 & $0.0405405405405$ & $4.62449086491$ \\
  \hline
  192.168.0.1 & 62 & $0.10472972973$ & $3.25525705524$ \\
  \hline
\end{tabular}\\

\vskip10pt

La entropía de la fuente es:

$$H(S) = 0.721963466885$$

En la figura \ref{fig:red1repliers:infoentro}, puede observarse gráficamente
los valores de información de cada símbolo en relación con la entropía de la
red.

\begin{figure}[h!]
    \centering                                                       
    \includegraphics[width=300pt]{respuestas1.png}
    \caption{Información de cada símbolo, en relación con la
        entropía de la red}
    \label{fig:red1repliers:infoentro}
\end{figure}

La figura \ref{fig:red1repliers:count} muestra la cantidad de apariciones de
cada símbolo.

\begin{figure}[h!]
    \centering
    \includegraphics[width=300pt]{red1repliers.png}
    \caption{Apariciones de los símbolos}
    \label{fig:red1repliers:count}
\end{figure}

En este caso, la IP 192.168.0.105 fue la que más veces contestó los pedidos
sobre su dirección MAC; de ahí su probabilidad tan alta. Como se puede
observar, la información aportada por este símbolo es considerablemente más
baja que la aportada por los demás que poseen menor probabilidad. Incluso, la
información que representa es más baja que la entropía de la fuente.
\newpage

\subsubsection{Símbolos: IPs por las que se realiza una consulta ARP}
La siguiente tabla muestra, para cada símbolo, su cantidad de apariciones y
sus medidas de probabilidad e información asociadas.

\vskip10pt

\begin{tabular}{|l|l|l|l|l|}
  \hline
  Símbolo & Consultas & Probabilidad & Información \\
  \hline
  169.254.37.204 & 3 & $0.00287081339713$ & $8.44432472625$\\
  \hline
  192.168.0.105 & 506 & $0.484210526316$ & $1.04629365227$\\
  \hline
  192.168.0.104 & 52 & $0.0497607655502$ & $4.32884750883$\\
  \hline
  169.254.255.255 & 10 & $0.00956937799043$ & $6.70735913208$\\
  \hline
  192.168.0.1 & 88 & $0.0842105263158$ & $3.56985560833$\\
  \hline
  192.168.0.101 & 30 & $0.0287081339713$ & $5.12239663136$\\
  \hline
  192.168.0.100 & 316 & $0.302392344498$ & $1.72550647879$\\
  \hline
  192.168.0.103 & 18 & $0.0172248803828$ & $5.85936222553$\\
  \hline
  192.168.0.102 & 22 & $0.0210526315789$ & $5.56985560833$\\
  \hline
\end{tabular}

\vskip10pt

La entropía de la fuente es:

$$H(S) = 1.99810125093$$

En la figura \ref{fig:red1requested:infoentro}, puede observarse gráficamente
los valores de información de cada símbolo en relación con la entropía de la
red.

\begin{figure}[h!]
    \centering                                                       
    \includegraphics[width=300pt]{consultadas1.png}
    \caption{Información de cada símbolo, en relación con la
        entropía de la red}
    \label{fig:red1requested:infoentro}
\end{figure}

La figura \ref{fig:red1requested:count} muestra la cantidad de apariciones de
cada símbolo.

\begin{figure}[h!]
    \centering
    \includegraphics[width=300pt]{red1requested.png}
    \caption{Apariciones de los símbolos}
    \label{fig:red1requested:count}
\end{figure}

En este caso, la IP más consultada en la red fue la 192.168.0.105. Al igual
que en el punto anterior, por consecuencia de su alta probabilidad de aparecer
en la fuente, su información no supera a la entropía que ofrece la red.
\newpage

\subsubsection{Símbolos: IPs a las que le respondieron una consulta ARP}
La siguiente tabla muestra, para cada símbolo, su cantidad de apariciones y
sus medidas de probabilidad e información asociadas.

\vskip10pt

\begin{tabular}{|l|l|l|l|l|}
  \hline
  Símbolo & Consultas & Probabilidad & Información \\
  \hline
  192.168.0.105 & 86.0 & 0.14527027027 & 2.78318861093\\
  \hline
  192.168.0.104 & 21.0 & 0.035472972973 & 4.81713594285\\
  \hline
  192.168.0.1 & 84.0 & 0.141891891892 & 2.81713594285\\
  \hline
  192.168.0.101 & 401.0 & 0.677364864865 & 0.561994939174\\
  \hline

\end{tabular}

\vskip10pt

La entropía de la fuente es:

$$H(S) = 1.35559706951$$

En la figura \ref{fig:red1replied:infoentro}, puede observarse gráficamente
los valores de información de cada símbolo en relación con la entropía de la
red.

\begin{figure}[h!]
    \centering                                                       
    \includegraphics[width=300pt]{respondidas1.png}
    \caption{Información de cada símbolo, en relación con la
        entropía de la red}
    \label{fig:red1replied:infoentro}
\end{figure}

La figura \ref{fig:red1replied:count} muestra la cantidad de apariciones de
cada símbolo.

\begin{figure}[h!]
    \centering
    \includegraphics[width=300pt]{red1replied.png}
    \caption{Apariciones de los símbolos}
    \label{fig:red1replied:count}
\end{figure}
\newpage

%%%%%%%%%%%%%%%%%%%%%%%%%%%%%%%%%%%%%%%%%%%%%%%%%%%%%%%%%%%%%%%%%%%%%%%
%*********************************************************************%
%*********************************************************************%
%******************************RED 2**********************************%
%*********************************************************************%
%*********************************************************************%
%%%%%%%%%%%%%%%%%%%%%%%%%%%%%%%%%%%%%%%%%%%%%%%%%%%%%%%%%%%%%%%%%%%%%%%

\subsection{Red 2}
\subsubsection{Símbolos: IPs que realizan una consulta ARP}
La siguiente tabla muestra, para cada símbolo, su cantidad de apariciones y
sus medidas de probabilidad e información asociadas.

\vskip10pt

\begin{tabular}{|l|l|l|l|l|}
  \hline
  Símbolo & Apariciones & Probabilidad & Información \\
  \hline
  192.168.0.105 & 57.0 & 0.0578093306288 & 4.11255382221\\
\hline
192.168.0.1 & 770.0 & 0.78093306288 & 0.356729200796\\
\hline
192.168.0.101 & 59.0 & 0.0598377281947 & 4.06280078702\\
\hline
192.168.0.100 & 3.0 & 0.00304259634888 & 8.36048133566\\
\hline
192.168.0.102 & 92.0 & 0.0933062880325 & 3.42188188032\\
\hline
0.0.0.0 & 5.0 & 0.00507099391481 & 7.62351574149\\
\hline

\end{tabular}

\vskip10pt

La entropía de la fuente es:

$$H(S) = 1.1428138485$$

En la figura \ref{fig:red2requesters:infoentro}, puede observarse gráficamente
los valores de información de cada símbolo en relación con la entropía de la
red.

\begin{figure}[h!]
    \centering                                                       
    \includegraphics[width=300pt]{red2/consultas2.png}
    \caption{Información de cada símbolo, en relación con la
        entropía de la red}
    \label{fig:red2requesters:infoentro}
\end{figure}

La figura \ref{fig:red2requesters:count} muestra la cantidad de apariciones de
cada símbolo. Podemos observar que la IP 192.168.0.1 es lavque más pedidos realiza.

\begin{figure}[h!]
    \centering                                                       
    \includegraphics[width=300pt]{red2/red2requesters.png}
    \caption{Apariciones de los símbolos}
    \label{fig:red2requesters:count}
\end{figure}

La figura \ref{fig:red2requesters:graph} permite observar qué IPs realizaron
consultas sobre qué otras IPs de la red. El tamaño de cada nodo se corresponde
con la cantidad de pedidos que realizó cada IP.

\begin{figure}[h!]
    \centering
    \includegraphics[width=350pt]{red2/red2requestersgraph.png}
    \caption{Grafo en el que hay un eje desde una IP A a una IP B si la IP A
        realizó una consulta por la IP B}
    \label{fig:red2requesters:graph}
\end{figure}


\newpage

\subsubsection{Símbolos: IPs que responden una consulta ARP}
La siguiente tabla muestra, para cada símbolo, su cantidad de apariciones y
sus medidas de probabilidad e información asociadas.

\vskip10pt

\begin{tabular}{|l|l|l|l|l|}
  \hline
  Símbolo & Apariciones & Probabilidad & Información \\
  \hline
  192.168.0.105 & 134.0 & 0.732240437158 & 0.449610647826\\
\hline
192.168.0.1 & 36.0 & 0.196721311475 & 2.34577483684\\
\hline
192.168.0.100 & 3.0 & 0.016393442623 & 5.93073733756\\
\hline
192.168.0.102 & 10.0 & 0.0546448087432 & 4.1937717434\\
\hline
\end{tabular}\\

\vskip10pt

La entropía de la fuente es:
$$H(S) = 1.11708005673$$

En la figura \ref{fig:red2repliers:infoentro}, puede observarse gráficamente
los valores de información de cada símbolo en relación con la entropía de la
red.

\begin{figure}[h!]
    \centering                                                       
    \includegraphics[width=300pt]{red2/respuestas2.png}
    \caption{Información de cada símbolo, en relación con la
        entropía de la red}
    \label{fig:red2repliers:infoentro}
\end{figure}

La figura \ref{fig:red2repliers:count} muestra la cantidad de apariciones de
cada símbolo.

\begin{figure}[h!]
    \centering
    \includegraphics[width=300pt]{red2/red2repliers.png}
    \caption{Apariciones de los símbolos}
    \label{fig:red2repliers:count}
\end{figure}


\newpage

\subsubsection{Símbolos: IPs por las que se realiza una consulta ARP}
La siguiente tabla muestra, para cada símbolo, su cantidad de apariciones y
sus medidas de probabilidad e información asociadas.

\vskip10pt

\begin{tabular}{|l|l|l|l|l|}
  \hline
  Símbolo & Consultas & Probabilidad & Información \\
  \hline
  192.168.0.105 & 134.0 & 0.135902636917 & 2.87935464592\\
\hline
192.168.0.1 & 51.0 & 0.051724137931 & 4.27301849441\\
\hline
192.168.0.101 & 86.0 & 0.0872210953347 & 3.51917908168\\
\hline
192.168.0.100 & 704.0 & 0.713995943205 & 0.486012217741\\
\hline
192.168.0.102 & 11.0 & 0.0111561866126 & 6.48601221774\\
\hline
\end{tabular}

\vskip10pt

La entropía de la fuente es:

$$H(S) = 1.33864665566$$

En la figura \ref{fig:red2requested:infoentro}, puede observarse gráficamente
los valores de información de cada símbolo en relación con la entropía de la
red.

\begin{figure}[h!]
    \centering                                                       
    \includegraphics[width=300pt]{red2/consultadas2.png}
    \caption{Información de cada símbolo, en relación con la
        entropía de la red}
    \label{fig:red2requested:infoentro}
\end{figure}

La figura \ref{fig:red2requested:count} muestra la cantidad de apariciones de
cada símbolo.

\begin{figure}[h!]
    \centering
    \includegraphics[width=300pt]{red2/red2requested.png}
    \caption{Apariciones de los símbolos}
    \label{fig:red2requested:count}
\end{figure}

\newpage

\subsubsection{Símbolos: IPs a las que le respondieron una consulta ARP}
La siguiente tabla muestra, para cada símbolo, su cantidad de apariciones y
sus medidas de probabilidad e información asociadas.

\vskip10pt

\begin{tabular}{|l|l|l|l|l|}
  \hline
  Símbolo & Consultas & Probabilidad & Información \\
  \hline
  192.168.0.105 & 49.0 & 0.267759562842 & 1.90098999417\\
\hline
192.168.0.1 & 67.0 & 0.366120218579 & 1.44961064783\\
\hline
192.168.0.101 & 50.0 & 0.273224043716 & 1.87184364851\\
\hline
192.168.0.102 & 17.0 & 0.0928961748634 & 3.42823699703\\
\hline
\end{tabular}

\vskip10pt

La entropía de la fuente es:

$$H(S) = 1.86964281144$$

En la figura \ref{fig:red2replied:infoentro}, puede observarse gráficamente
los valores de información de cada símbolo en relación con la entropía de la
red.

\begin{figure}[h!]
    \centering                                                       
    \includegraphics[width=300pt]{red2/respondidas2.png}
    \caption{Información de cada símbolo, en relación con la
        entropía de la red}
    \label{fig:red2replied:infoentro}
\end{figure}

La figura \ref{fig:red2replied:count} muestra la cantidad de apariciones de
cada símbolo.

\begin{figure}[h!]
    \centering
    \includegraphics[width=300pt]{red2/red2replied.png}
    \caption{Apariciones de los símbolos}
    \label{fig:red2replied:count}
\end{figure}


%%%%%%%%%%%%%%%%%%%%%%%%%%%%%%%%%%%%%%%%%%%%%%%%%%%%%%%%%%%%%%%%%%%%%%%
%*********************************************************************%
%*********************************************************************%
%***************************CONCLUSION********************************%
%*********************************************************************%
%*********************************************************************%
%%%%%%%%%%%%%%%%%%%%%%%%%%%%%%%%%%%%%%%%%%%%%%%%%%%%%%%%%%%%%%%%%%%%%%%
\newpage
\section{Conclusión}
A partir de los datos obtenidos se puede observar que hay más $requests$ que
$replys$, ésto se debe entre otras cosas a que los hosts hacen consultas por
su propia IP para asegurarse que no haya otro host con ésa dirección en la
red. A su vez puede haber consultas que no son respondidas y paquetes que se
pierden.\\ 
También se desprende de los datos que el router (192.168.0.1 en ambas redes analizadas)
es siempre uno de los nodos más activos de la red. Ésto tiene mucho sentido ya que la mayor parte del tráfico de la red pasa por él.\\
Nos resultó extraño que apareciera como IP fuente de algunos
pedidos la IP 0.0.0.0 pero investigando un poco nos dimos cuenta que ésto
sucede cuando un host se conecta a la red y se hace para verificar que no haya
otro host con la misma IP.\\
Otra conclusión que pudimos sacar, no tanto de los datos sino de trabajar con el protocolo
es que éste no es seguro ya que permite que un host se haga pasar por otro,
respondiendo mensajes ARP que no estaban dirigidos a él. Ésto se conoce como ARP Spoofing.

\end{document}
